  \documentclass{beamer}
  \usepackage{graphicx}
  \usetheme{Madrid}

  % Title page information
  \title{COMP 354 Iteration 1 Eternity Project}
  
  \subtitle {Team I}

  \author
  {
  Hong Phuc Nguyen \\ Swati Pareek \\Anik Patel\\Avnish Patel 
  \\Clément Potteck \\ Tara Seal \\Emanuel Sharma
  }

  \institute{
  Concordia University
  }
  \date{Friday June 5, 2020}

  \begin{document}

  \frame{\titlepage}

  % wondering if the TOC is necessary for a 10-min presentation?
  % I would agree about title pages separating main sections though
  \begin{frame}
  \frametitle{Overview}
  \tableofcontents
  \end{frame}

  \section{Team}


  
  \begin{frame}
  \frametitle{The Team}
  \null\hfil\hfil\makebox[2cm]{Nguyen}
  \hfil\hfil\makebox[2cm]{Swati}
  \hfil\hfil\makebox[2cm]{Anik}
  \hfil\hfil\makebox[2cm]{Avnish}
  \newline
  \hfil\includegraphics[width=2cm]{Screen Shot}
  \hfil\hfil\hfil\includegraphics[width=2cm]{Screen Shot}
  \hfil\hfil\hfil\includegraphics[width=2cm]{Screen Shot}
  \hfil\hfil\hfil\hfil\includegraphics[width=2cm]{Screen Shot}\newline
  \null\hfil\hfil\makebox[2cm]{MAD}
  \hfil\hfil\makebox[2cm]{\[e^x\] XOR \[a^x\]}
  \hfil\hfil\makebox[2cm]{sin(x) XOR cos(x)}
  \hfil\hfil\makebox[2cm]{sinh(x)}
  \newline\newline
  \vfil
  \null\hfil\hfil\makebox[2cm]{Clément}
  \hfil\hfil\makebox[2cm]{Tara}
  \hfil\hfil\makebox[2cm]{Emanuel}\newline
  \hfil\hfil\includegraphics[width=2cm]{Screen Shot}
  \hfil\hfil\includegraphics[width=2cm]{Screen Shot}
  \hfil\hfil\hfil\includegraphics[width=2cm]{Screen Shot}\newline
  %There's an error about the dollar sign with these functions
  \null\hfil\hfil\makebox[2cm]{\[10^x\] XOR \[\Pi^x\]}
  \hfil\hfil\makebox[2cm]{\[x^y\]}
  \hfil\hfil\makebox[2cm]{ln(x) XOR log_1_0(x) }\newline
  \end{frame}
  
  
  
    \begin{frame}
  \frametitle{Team Roles}
  \small
  \begin{table}
\begin{tabular}{l | c | c | c | c }
Name & Primary Role and Responsibility & Secondary Role \\
\hline
Hong Phuc & GUI and Web application,& Implementation of\\
& prototyping & new technologies\\
 \hline
Swati & Organizing and planning agenda& High-level vision, \\
& for team meetings & scope management\\
 \hline
Anik & Set up of initial repository& Questions regarding\\
& GUI for local application & the Python ecosystem\\
 \hline
Avnish & Technical Writer & Mathematics, \\
& & documentation\\
 \hline
Clément & Major presenter, & Best practices, \\
& quality control  & PEP documentation\\
 \hline
Tara & Minor presenter, & Ensuring requirement \\
& Team liason  & are followed\\
 \hline
Emmanuel & Technical writer, &  Algorithm optimization\\
& subject matter expert  & latex\\
& for math algorithms  & \\
\end{tabular}
\end{table}
  \end{frame}

  \begin{frame}
  \frametitle{Collaboration patterns:\\ Managing the project}
  \begin{itemize}
   \item Regular meetings once a week
   \item Agenda
     \begin{itemize}
     \item Keeps meetings focused and on track
     \item Meeting transcriber ensures that discussions are on track with agenda
     \end{itemize}
   \item Questions and Answers
      \begin{itemize}
     \item Questions are gathered by the meeting transcriber
     \item Questions are sent to the professor
     \end{itemize}
   \item Status Update: done, doing, to-do
   \item Follow up on decisions and commitments
  \end{itemize}
  \end{frame}

  \begin{frame}
  \frametitle{Collaboration Patterns: \\ Centralizing work product management}
  \begin{itemize}
   \item Discord
    \begin{itemize}
     \item Group channels: chat, pings, pins
     \item Direct Messages: on-on-one conversations
     \item integrated voice calls for meetings
    \end{itemize}
   \item Google Drive
   \item GitHub
    \begin{itemize}
     \item Quality Control: protected branches
       \begin{itemize}
          \item protected branches
          \item staging branch
          \item master branch
      \end{itemize}
     \item Required asymmetric code reviews
    \end{itemize}
  \end{itemize}
  \end{frame}


  \begin{frame}
  \frametitle{Asymmetric review}
  \begin{itemize}
   \item One individually responsible for conducting the code review
   \item Additional technique used:
    \begin{itemize}
     \item Available for comments by anyone in the team
     \item Allows for skill utilization
    \end{itemize}
  \end{itemize}
  \end{frame}



  \section{Requirement Gathering}

  % Slide for the interview questions
  \begin{frame}
  \frametitle{Interview Questions}
  \begin{itemize}
   \item Demographic of people who use a calculator
    \begin{itemize}
     \item Professional/educational background
     \item Calculator Usage
    \end{itemize}
   \item User needs
    \begin{itemize}
     \item Degree of precision
     \item Input box vs. button selection
     \item Numeral system required
    \end{itemize}
   \item Separate requirements into needs and preferences
    \begin{itemize}
     \item Ease of use
     \item Aesthetic
     \item Features
     \item Platform
    \end{itemize}
     \item Frustrations with current device
  \end{itemize}
  \end{frame}


  %Slide for the interview model
  \begin{frame}
  \frametitle{Interview Model}
  \begin{itemize}
   \item Hourglass
    \begin{itemize}
     \item General questions asking the user to describe what they do and how they use a calculator
     \item Specific questions related to their work
     \item Ended interview with an open-ended question to see if there is anything they'd like to add
    \end{itemize}
   \item Findings
    \begin{itemize}
     \item Many interviewees went back to previous questions we asked during the open-ended question at the end to clarify something they said.
    \end{itemize}
  \end{itemize}
  \end{frame}


  %Slide for who we interviewed
  \begin{frame}
  \frametitle{Interviee Demographics}
  \begin{itemize}
   \item Students
    \begin{itemize}
     \item High school
     \item Cégep
     \item University
    \end{itemize}
   \item Researchers
  \begin{itemize}
   \item Pathologist
      \item Animal Biologist
    \end{itemize}
  \end{itemize}
  \end{frame}
  
  
    %Slide for our interview findings
  \begin{frame}
  \frametitle{Interview: Key findings}
  \begin{itemize}
   \item Precision is extremely important
   \item Existing calculators contain a lot of unnecessary buttons and features
   \item Split findings on local or web application
    \begin{itemize}
     \item Mobility when conducting experiments or for exams
     \item Preference for desktop applicaton when working at computer
    \end{itemize}
  \end{itemize}
  \end{frame}
  
%Slide for our interview findings
  \begin{frame}
  \frametitle{Interview: Other interesting findings}
  \begin{itemize}
   \item Speech recognition
   \item Entering numbers without a numpad on a laptop creates challenges
       \begin{itemize}
     \item Should have the ability to both click buttons and input numbers
    \end{itemize}
   \item Functionality to keep track of history
   \item Ability to keep calculator open as a top window at all times on the computer
  \end{itemize}
  \end{frame}
  

%Slide for our interview findings
  \begin{frame}
  \frametitle{Summarized Use case}
  \includegraphics[scale=0.5]{Use Case}
  \end{frame}


  \begin{frame}
  \frametitle{Inclusions/Exclusions}
  \begin{itemize}
   \item Information from users gave us a clear idea on what should be built
    \begin{itemize}
    \item Some ideas were not feasible in the timeframe
    \begin{itemize}
      \item Speed recognition
      \item Radian and degree conversions
    \end{itemize}
    \item Scope \& reasoning
      \begin{itemize}
      \item Essential features/functionality - for iteration 1
      \item Features/functionality to be looked at a later point
    \end{itemize}
    \end{itemize}
    \end{itemize}
  \end{frame}


  %Slide for future directions
  \begin{frame}
  \frametitle{Future directions for iteration 2}
  \begin{itemize}
   \item Features
    \begin{itemize}
     \item Optimization
     \item Speech recognition
    \end{itemize}
   \item Collaboration techniques
    \begin{itemize}
     \item Buddy system
     \item Role rotation
    \end{itemize}
  \end{itemize}
  \end{frame}



\section{Product}
\begin{frame}
\frametitle{Sample title2}
This is a text in the first frame. This is a text in the first frame. This is a text in the first frame.
\includegraphics[width=3cm, height=4cm]{Screen Shot}
\end{frame}

  \end{document}
