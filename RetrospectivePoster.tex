% Please note that this template was taken from the following URL:
% https://www.latextemplates.com/template/dreuw-deselaers-poster

\documentclass[final,hyperref={pdfpagelabels=false}]{beamer}
\usepackage[orientation=landscape,size=a0,scale=1.4]{beamerposter}
\usetheme{I6pd2} % Use the I6pd2 theme supplied with this template


% \usepackage{times}\usefonttheme{professionalfonts}  % Uncomment to use Times as the main font



%-----------------------------------
%	TITLE SECTION 
%-----------------------------------
\title{\huge Retrospective Poster} % Poster title
\author{Team I} % Author(s)
\institute{COMP 354: Introduction to Software Engineering} % Institution(s)

% %-----------------------------------
% % %	FOOTER TEXT
% %-----------------------------------

% \newcommand{\leftfoot}{Team I} % Left footer text

% \newcommand{\rightfoot}{COMP 354} % Right footer text

%-----------------------------------
\begin{document}

\addtobeamertemplate{block end}{}{\vspace*{2ex}} % White space under blocks

\begin{frame}[t] % The whole poster is enclosed in one beamer frame

\begin{columns}[t] % The whole poster consists of two major columns, each of which can be subdivided further with another \begin{columns} block - the [t] argument aligns each column's content to the top

\begin{column}{.02\textwidth}\end{column} % Empty spacer column

\begin{column}{.465\textwidth} % The first column

%----------------------------------------------------------------------------------------
%	OBJECTIVES
%----------------------------------------------------------------------------------------

\begin{block}{Purpose}

\begin{itemize}
   \item Team motivation 
   \begin{itemize}
     \item See how everyone enjoyed their roles in iteration 1, and if they would prefer different roles for iteration 2
     \item Understand everyone's expectation for iteration 2
     \item Share insight into what everyone has learned from each other
   \end{itemize}
   
   \item Team strengths and weaknesses
   \begin{itemize}
     \item Understand how everyone thought we worked together
     \item What did we do well that we should continue for iteration 2?
     \item What didn't we do well that should be improved for iteration 2?
   \end{itemize}

   \item Evaluation of tools used
   \begin{itemize}
     \item Tools/techniques that worked 
     \item Tools/techniques that did not work
     \item Possible future tools that should be used that might help with collaboration/organization
   \end{itemize}
\end{itemize}
\end{block}

%----------------------------------------------------------------------------------------
%	Methodology
%----------------------------------------------------------------------------------------
            
\begin{block}{Methodology}
\begin{itemize}
\item To conduct the retrospective, everyone was individually asked the questions below. 
\begin{enumerate}
   \item What are your expectations of the team for the project? Have they changed since the first delivery? 
   \item What are your weaknesses and/or tasks that make you uncomfortable?
   \item Do you want more tasks to help you overcome them or would you prefer to have tasks corresponding to your strengths? 
   \item What are some things we need to improve for our team dynamic? 
   \item What was your most frustrating experience in this project?
   \item Describe an instance that another team member has inspired you. What did you learn? 
   \item Did you enjoy your role in Iteration 1? Why or Why not?
   \item Would you like to contribute differently (i.e. different role, more/less programming, more/less writing) in this iteration, and if so how? 
   \item If you needed to redo iteration 1 would there be anything you would do differently?
\end{enumerate}
\item After everyone individually answered the questions, we discussed the results as a team, so that everyone within the team was aware of each other's responses and suggestions.
\end{itemize}
\end{block}


\end{column} % End of the first column

\begin{column}{.03\textwidth}\end{column} % Empty spacer column
 
\begin{column}{.465\textwidth} % The second column


%----------------------------------------------------------------------------------------
%	Team dynamic findings
%----------------------------------------------------------------------------------------

\begin{block}{General Findings}

\begin{itemize}
\item Overall, the majority of people's expectations for iteration 1 were met.
\item The team felt as if the majority of people were adhering to deadlines
\item People were good at communicating and responding, allowing for the difficulties with not meeting face-to-face being minimized.
\item Individuals felt as if people were always willing to help when someone was unsure how to do something.
\item Some people were initially intimidated by certain technologies that team members were using but realized that many of the team members were happy to help if questions were asked.
\item Some individuals felt as if communication could be improved. For example, if someone cannot attend a meeting, that the team should be informed. Sometimes this was not the case and the meeting attendees were only discovered during the meeting itself.
\item Most people agreed that it was a bit rushed in the last days, and that although we were proactive, we should try to find ways to avoid this.
\end{itemize}
\end{block}







%----------------------------------------------------------------------------------------
%	Roles
%----------------------------------------------------------------------------------------
            
\begin{block}{Iteration 2 Roles}
\begin{itemize}
\item Everyone seemed happy with the role they undertook during iteration 1. 
\item For iteration 2, some team members wanted to keep their role, while others wanted to try different roles.
\item Based off this feedback, we determined the best roles for everyone for iteration 2.
\end{itemize}
\end{block}




%------------------------------------------------

\begin{block}{Iteration 2 tool suggestions}
\begin{itemize}
\item Kanban board
    \begin{itemize}
        \item This would enable us to have better visibility on what everyone is working on.
        \item If the team is aware on what everyone is working on, other team members might also be able to help.
    \end{itemize}
\item Automation Testing
    \begin{itemize}
        \item This would enable us to conduct more thorough testing.
    \end{itemize}

\end{itemize}
\end{block}









\end{column} % End of the second column

\begin{column}{.015\textwidth}\end{column} % Empty spacer column

\end{columns} % End of all the columns in the poster



% %--------------------------
% %	Conclusion
% %--------------------------


  
  
\begin{block}{Conclusion}
\begin{itemize}
\item The retrospective enabled our team to understand how everyone felt we did during iteration 1, and allowed us to come up with suggestions on how we can improve for iteration 2
\item Some areas of communication we thought could be improved were mentioned, though overall the team thought that we worked well. 
\item In order to minimize the last-minute stress that was encountered during iteration 1, the technical writing portion should be started earlier
\end{itemize}
\end{block}




\end{frame} % End of the enclosing frame

\end{document}